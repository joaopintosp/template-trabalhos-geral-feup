%! suppress = EscapeHashOutsideCommand
%! suppress = MissingLabel
\documentclass[11pt,a4paper]{report}
\usepackage[portuguese]{babel}
\usepackage{indentfirst}
\usepackage{mathptmx}
\usepackage[pt]{datetime2}
\usepackage[utf8]{inputenc}
\usepackage{csquotes}
\usepackage{slantsc}
\usepackage[version=4]{mhchem}
%%%%%---------------------ESPAÇAMENTO E AVANÇO DE PARÁGRAFOS-------------------%%%%%%%%%
\usepackage{setspace}
\usepackage{emptypage}
\onehalfspacing
\setlength{\parindent}{1.25cm}
\setlength{\parskip}{6pt}
\newcommand{\newpara}
{
    \vskip 6pt
}

%%%%%--------------------------------------------------------------------------%%%%%%%%%
% Para gerar referências bibliográficas
\usepackage[style=apa]{biblatex}
\addbibresource{main.bib}

\usepackage{amsmath} % Para fórmulas matemáticas
\usepackage{hyperref, bookmark} % Para gerar os Índices
\hypersetup{pdfborder=0 0 0}
\usepackage{lipsum} % Para gerar texto aleatório
\usepackage{booktabs}


%%%%%-----------------------Referir-Figuras/Tabelas...---------------------------%%%%%%%%%
\usepackage[capitalize, brazilian]{cleveref}
% "cleveref" Portuguese names.
\Crefname{equation}{Equação}{Equações}
\Crefname{table}{Tabela}{Tabelas}
\Crefname{figure}{Figura}{Figuras}
\crefname{chapter}{capítulo}{capítulos}
\Crefname{chapter}{Capítulo}{Capítulos}
\crefname{section}{secção}{secções}
\Crefname{section}{Secção}{Secções}
%%%%%--------------------------------------------------------------------------%%%%%%%%%

%%%%%--------------------------------MARGENS-----------------------------------%%%%%%%%%
\RequirePackage[outer=25mm,inner=35mm,vmargin=20mm,includehead,includefoot,headheight=15pt]{geometry}
\setlength{\headheight}{14pt}

%%%%%--------------------------------------------------------------------------%%%%%%%%%

%%%%%---------------------------PERSONALIZAR TABELAS---------------------------%%%%%%%%%
\usepackage{array, multirow,multicol,tabularx}
\usepackage[table]{xcolor}
\newcolumntype{S}{>{\columncolor{black!20}} c}
\newcolumntype{Y}{>{\columncolor{yellow!20}} c}
\renewcommand{\arrayrulewidth}{0.2mm}
\newcolumntype{R}{>{\columncolor{red!30}} c}
\usepackage{caption}
\captionsetup[table]{skip=2pt}

%%%%%--------------------------------------------------------------------------%%%%%%%%%
\usepackage{graphicx} % Para introduzir imagens no documento
\graphicspath{{imagens/}} % Definir a pasta das imagens

%%%%%-------------------------Gráficos e Plots 2D ou 3D------------------------%%%%%%%%%
\usepackage{tikz,pgfplots}
\usetikzlibrary{positioning}

%%%%%----------------------------CABEÇALHO E RODAPÉ----------------------------%%%%%%%%%
%%%%%---------------------------------ESTILO 1---------------------------------%%%%%%%%%
% \usepackage{fancyhdr}
% \renewcommand{\chaptermark}[1]{\markboth{#1}{}}
% \fancypagestyle{plain}{%  first page of chapters
%     \fancyhf{} % clear all header and footer fields
%     \fancyfoot[C]{\thepage}
%     \renewcommand{\headrulewidth}{0pt} % no rule
%     \renewcommand{\footrulewidth}{0pt} % Adicionar Linha no rodapé (Predefinição 0pt)
% }

% \fancypagestyle{fancy}{%  all pages
%     \fancyhf{} % clear all header and footer fields
%     \fancyhead[L]{\leftmark}
%     \fancyhead[R]{Unidade Curricular}
%     \fancyfoot[R]{\thepage}
%     \fancyfoot[L]{João Sá Pereira}
%     \renewcommand{\headrulewidth}{0.4pt}
% \renewcommand{\footrulewidth}{0.4pt}
% }

%%%%%--------------------------------------------------------------------------%%%%%%%%%

%%%%%----------------------------CABEÇALHO E RODAPÉ----------------------------%%%%%%%%%
%%%%%---------------------------------ESTILO 2---------------------------------%%%%%%%%%
\usepackage{fancyhdr}
\renewcommand{\chaptermark}[1]{\markboth{#1}{}}
\fancypagestyle{plain}{%  first page of chapters
    \fancyhf{} % clear all header and footer fields
    \fancyfoot[C]{\thepage}
    \renewcommand{\headrulewidth}{0pt} % no rule
    \renewcommand{\footrulewidth}{0pt} % Adicionar Linha no rodapé (Predefinição 0pt)
}

\fancypagestyle{fancy}{%  all pages
    \fancyhf{} % clear all header and footer fields
    \fancyhead[L]{\textit{\leftmark}}
    \fancyhead[R]{\thepage}
    \renewcommand{\headrulewidth}{0pt}
    \renewcommand{\footrulewidth}{0pt}
}
\pagestyle{fancy} % apply the stye <<<<<<<<<<<<<

\usepackage[explicit,compact]{titlesec}
\titleformat{\chapter}[block]
{\bfseries\huge}{\filright\huge\thechapter.}{1ex}{\huge\filright #1}
\titlespacing*{\chapter}{0pt}{-30pt}{30pt}
\titlelabel{\thetitle.\quad}
\usepackage{subcaption}
\usepackage{hhline}
%%%%%--------------------------------------------------------------------------%%%%%%%%%
\usepackage{fontspec}
\setmainfont{Calibri}
\setcounter{MaxMatrixCols}{20}
\usepackage{adjustbox}
\usepackage{float}
\usepackage{enumitem}
\newcommand{\graus}{$^{\circ}$C }
%%%%%--------------------------------------------------------------------------%%%%%%%%%
%%%%%----------------------------ÍNICIO DO DOCUMENTO---------------------------%%%%%%%%%
%%%%%--------------------------------------------------------------------------%%%%%%%%%
\pagenumbering{roman}
\begin{document}
    \begin{titlepage}
    \setlength{\headheight}{14pt}
    \begin{center}
        \large
        \Large\textbf{{\scshape Faculdade de Engenharia da Universidade do Porto}}
        \vspace*{1.8cm}

        \huge
        \textbf{Título}\\
        \textbf{Subtitulo} \\
        \vspace{0.5cm}
        \LARGE
        Unidade Curricular

        \vspace{1cm}
        \Large
        \textbf{João Pinto dos Santos Sá Pereira}

        \vspace{3cm}

        \includegraphics[width=0.5\textwidth]{imagens/uporto-feup.pdf}
        \vspace{2cm}

        \large
        Mestrado em Engenharia de Minas e Geo-Ambiente\\
        \vspace{0.5cm}
        Docentes:\\
        Nome do Professor
        \vfill
        \today

    \end{center}
\end{titlepage}

    \shipout\null

%%%%%---------------------------------PREÂMBULO--------------------------------%%%%%%%%%
    \cleardoublepage
    \chapter*{Resumo}
\lipsum[1]
    \cleardoublepage
    \pdfbookmark[0]{Conteúdo}{contents}
    \tableofcontents
    \cleardoublepage
    \pdfbookmark[0]{Lista de Figuras}{figures}
    \listoffigures
    \cleardoublepage
    \pdfbookmark[0]{Lista de Tabelas}{tables}
    \listoftables
    \cleardoublepage
%%%%%--------------------------------------------------------------------------%%%%%%%%%

%%%%%-----------------------------CORPO DO DOCUMENTO---------------------------%%%%%%%%%
    \pagenumbering{arabic}
    \chapter{Isto é apenas um teste.}

\lipsum[1]

\section{Continuação do teste}

\lipsum[2-3]
%    \include{chapters/chapter2}
%    \include{chapters/chapter3}
%    \include{chapters/chapter4}
%%%%%--------------------------------------------------------------------------%%%%%%%%%

%%%%%--------------------------BIBLIOGRAFIA E ANEXOS---------------------------%%%%%%%%%
    %\nocite{leite_diagramas,leite_modelacao} %%---> Para criar bibliografia com as referências não citadas.
    %\printbibliography % Gera as Referências Bibliográficas

%%%%%--------------------------------------------------------------------------%%%%%%%%%
% \appendix
% \chapter{Caso de Estudo}

\section{Parâmetros físico-químicos medidos nas amostras recolhidas} \label{ap:parametros}


%%%%%--------------------------------------------------------------------------%%%%%%%%%
\end{document}
